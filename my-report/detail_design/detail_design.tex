\graphicspath{{detail_design/fig/}}

\chapter{Detail Designs}
\label{chap:detail_design}

\section{Micro-controller}
% discuss how all functional blocks connect and communicate with each other 
The Adafruit STM32F405 feather has a supply voltage of 3.7V and consumes up to $80mA$ when operational, it is powered by a 5000mAh 3.7V lithium polymer 
battery connected with a jst connection. The board features a 3.3V regulator which is to regulate the battery power, there are also pins to distribute 
the 3.3V regulated power off-board. The 3.3V regulator is used to power the GPS receiver, MPU-9250/6500 IMU, and the micro-SD socket. The board has a USB-C 
female connector which is used to load the main program and for debugging during the design process. The board is capable of running circuit python, which was
used in the development of this project. 
%Not sure if i should mention this :(


\section{GPS receiver}
The GPS receiver has a power supply voltage of 3.3V and is powered by the Feathers 3.3V regulator. It is connected to a pin on the feather which is 
configured as UART, with a BAUD rate of 9600. The GPS receiver continually receives data at a frequency which is set during the configuration process, in 
this case its every second. Data sent by the GPS receiver is in the form of ASCII encoded sentences which follow the NMEA protocol\cite{nmea}. The NMEA 
protocol defines a number of ASCII identifiers which occur at the beginning of each ASCII sentence, they are used to identify the type of data 
contained in the sentence. Table \ref{table:NMEA} shows the specific NMEA sentences which are relevant to this project and their corresponding ASCII 
identifiers.
\\

\begin{table}[!h]
    \centering
    \caption{NMEA sentence information}
    \label{table:NMEA}
    \begin{tabular}{ | m{10em} | m{25em} | }
        
        \hline
        \textbf{Sentence identifier} & \textbf{Description} \\
        \hline
        \$GPGGA & Global Positioning System Fix Data (Time, Latitude, Longitude) \\
        \hline
        \$GPGLL & Geographic Position, Latitude / Longitude and time. \\
        \hline
        \$GPGSV & GPS Satellites in view \\
        \hline
        \$GPVTG & Track Made Good and Ground Speed. \\
        \hline
        \$GPRMC & Recommended minimum specific GPS/Transit data (Latitude, Longitude, Speed over ground in knots, Track angle made true, Magnetic variation) \\
        \hline
    \end{tabular}
\end{table}

% Not sure it uses PMTK
% The GPS receiver can be configured by sending the appropriate PMTK command packets\cite{pmtk}. These commands are the same for all GPS receivers. The GPS receiver was
% set to send GGA and RMC NMEA sentences, and the update rate was set to $2Hz$ by sending the appropriate PMTK commands.

\section{Wind Direction Sensor}
The wind direction sensor has a supply voltage in the range 12~30VDC, and is powered by two 9V alkaline batteries in series. It outputs a $4-20mA$ analog signal,
which is connected directly to pin A4 of the Feather. Pin A4 on the Feather connects to a 5V tolerant pin on the STM32F405 micro-controller. This pin was 
configured to use one of the micro-controllers 12-bit ADC, which has a sampling range of $0V$ to $3.3V$. A shunt resistor with a value of $160\Omega$ is 
connected to the sensor output. This value was calculated to produce a voltage over the resister in the range of $642mV$ to $3.2V$. The Actual voltage range
measured over the shunt resister was $642mV$ to $3.07V$.

All pins on the STM32F405 micro-controller are capable of sinking or sourcing $25mA$ maximum, there is therefore no chance of the sensor damaging the micro-controller
pin. A digital buffer was initially considered to prevent the micro-controller from loading the sensor output circuit, but was not used as the ADC inputs are
high impedance. The STM32F405 data sheet\cite{STM32F405} specifies that for an ADC input, the max external impedance allowed for an error below $1/4$ of the LSB as $50K\Omega$. 
The shunt resistors value is multitudes smaller than this.     

\section{Servo Motors}
Both the rudder servo motor and the sail servo winch are powered by a 6V source consisting of four 1.5V alkaline batteries. The rudder servo motor is connected to pin
A2 of the feather, the Sail winch servo motor is connected to pin D9 of the Feather. Both these pins are configured to output a 50HZ PWM signal, with a duty cycle set
by the rudder and sail position controllers. The rudder angle is measured from the centre line of the boat, this corresponds to the neutral position of the rudder. Positive 
rudder angle is measured from neutral position to the starboard side of the vessel, negative rudder angle is measured from the neutral position to port side of the vessel. 
Sail position is measured in degrees from the centre line of the vessel to either the port or starboard side. The maximum and minimum PWM values for both servo motors was 
determined experimentally and is summarized in table \ref{table:pwm}, along with the corresponding angles. 

\begin{table}[!h]
    \centering
    \caption{Servo motor PWM limits}
    \label{table:pwm}
    \begin{tabularx}{\columnwidth}{ | X | X | X | X | X | }
        
        \hline
        Servo motor & PWM\_Min & PWM\_Min corresponding angle & PWM\_Max & PWM\_Max corresponding angle \\
        \hline
        Rudder & 48 & $60^{\circ}$ & 102 & $-60^{\circ}$ \\
        \hline
        Sail & 72 &  $0^{\circ}$ & 120 & $80^{\circ}$\\
        \hline
    \end{tabularx}
\end{table}

The PWM duty cycle value is stored in a 16 bit register and is determined using Eq. \ref{eq:duty_cycle}.

\begin{equation}
    \label{eq:duty_cycle}
    duty\_cycle = (\frac{PWM\_value}{1000}) \cdot 2^{16}
\end{equation}

\section{Micro-SD card}
The micro-SD card socket has a supply voltage of 3.3V and is powered with the Feathers 3.3V regulator. It is connected to the Feather using a SPI interface. The micro-SD card sockets MOSI, MISO,
 and clock lines are connected directly to the corresponding pins on the Feather, the chip select line is connected to pin 6 of the Feather. The micro-SD card is formatted with the FAT32 file 
 system, and as, such writing to the micro-SD card is done with FAT32 protocol.


\section{Digital compass}
The MPU-9250/6500 IMUhas a supply voltage of 3.3V and is powered by the Feather boards 3.3V regulator. It is connected to the Feather via a $I^{2}C$ connection. The IMU has an address pin 
which is grounded. The purpose of the address pin is for setting the LSB of the $I^{2}C$, this allows two IMU's to be connected to the same bus, which is not needed in this case. 

\subsection{Tilt Compensation}
To perform tilt compensation, firstly pitch and roll angles of the inertial frame of the vessel must be found. Eq. \ref{eq:acc_1} is used to transform a sampled acceleration vector back to 
the NED frame of reference, where there exists only a z-component with magnitude 9.81 $m/s^2$. In this eqution $\vec{A}_s$ represents the acceleration vector which has been 
sampled from the IMU, g is acceleration due to gravity and $\vec{R_{x,\phi}}$, $\vec{R_{y,\theta}}$, $\vec{R_{z,\psi}}$ represent equations \ref{eq:R_x}, \ref{eq:R_y}, \ref{eq:R_z} 
respectively. 

\begin{equation}
    \label{eq:acc_1}
    \vec{R_{y,\theta}} \cdot \vec{R_{x,\phi}} \cdot \vec{A_s} = \begin{bmatrix} 0 & 0 & g \end{bmatrix}^T
\end{equation}

Expanding Eq. \ref{eq:acc_1}, gives Eq. \ref{eq:acc_1_expand}. The components of the sampled acceleration vector are $A_{sx}$, $A_{sy}$ and $A_{sz}$.

\begin{equation}
    \label{eq:acc_2}
    \begin{bmatrix} cos\theta & 0 & sin\theta \\ 0 & 1 & 0 \\ -sin\theta & 0 & cos\theta \end{bmatrix} \begin{bmatrix} 1 & 0 & 0 \\ 0 & cos\phi & -sin\phi \\ 0 & sin\phi & cos\phi \end{bmatrix} \begin{bmatrix} A_{sx} \\ A_{sy} \\ A_{sz} \end{bmatrix}= \begin{bmatrix} 0 \\ 0 \\ g \end{bmatrix}
\end{equation}

Expanding Eq.\ref{eq:acc_2} further gives Eq.\ref{eq:acc_3},then Eq. \ref{eq:acc_4} and \ref{eq:acc_5}.

\begin{equation}
    \label{eq:acc_3}
    \begin{bmatrix} cos\theta & sin\theta sin\phi & sin\theta cos\phi \\ 0 & cos\phi & -sin\phi \\ -sin\theta & cos\theta sin\phi & cos\theta cos\phi \end{bmatrix} \begin{bmatrix} A_{sx} \\ A_{sy} \\ A_{sz} \end{bmatrix}= \begin{bmatrix} 0 \\ 0 \\ g \end{bmatrix}
\end{equation}

\begin{equation}
    \label{eq:acc_4}
    A_{s,x}cos\theta + A_{s,y}sin\theta sin\phi + A_{s,z}sin\theta cos\phi = 0
\end{equation}

\begin{equation}
    \label{eq:acc_5}
    A_{s,y}cos\phi - A_{s,z}sin\phi = 0
\end{equation}

Roll angle $\phi$ can be found using Eq.\ref{eq:roll} which results from rearranging terms in Eq.\ref{eq:acc_5}

\begin{equation}
    \label{eq:roll}
    \phi = \arctan (\frac{A_{s,y}}{A_{s,z}})
\end{equation}

By substituting roll angle $\phi$ into Eq.\ref{eq:pitch} - which is derived from Eq.\ref{eq:acc_4} - pitch angle $\theta$ is found.

\begin{equation}
    \label{eq:pitch}
    \theta = \arctan( \frac{ -A_{s,x}}{A_{s,y}sin\phi + A_{s,z}cos\phi})
\end{equation}

Now that pitch and roll angles of the inertial frame of the vessel are known, a sampled magnetometer vector - $\vec{B_{s}}$ - can be transformed to the NED frame of reference. This is done
using Eq.\ref{eq:mag_1}, where $\vec{B_{r}}$ is the transformed vector. Eq.\ref{eq:R_z} is not used here as this will eliminate yaw angle $\psi$ whixh in needed to determine the resulting bearing

\begin{equation}
    \label{eq:mag_1}
    \vec{B_{r}} = \vec{R_{y,\theta}} \cdot \vec{R_{x,\phi}} \cdot \vec{B_{s}}
\end{equation}

Eq.\ref{eq:mag_1} is then expanded, giving Eq.\ref{eq:mag_2} and Eq.\ref{eq:mag_3}.

\begin{equation}
    \label{eq:mag_2}
    \begin{bmatrix} B_{r,x} \\ B_{r,y} \\ B_{r,z} \end{bmatrix} = \begin{bmatrix} cos\theta & 0 & sin\theta \\ 0 & 1 & 0 \\ -sin\theta & 0 & cos\theta \end{bmatrix} \begin{bmatrix} 1 & 0 & 0 \\ 0 & cos\phi & -sin\phi \\ 0 & sin\phi & cos\phi \end{bmatrix} \begin{bmatrix} B_{s,x} \\ B_{s,y} \\ B_{s,z} \end{bmatrix}
\end{equation}

\begin{equation}
    \label{eq:mag_3}
    \begin{bmatrix} B_{r,x} \\ B_{r,y} \\ B_{r,z} \end{bmatrix} = \begin{bmatrix} B_{s,x} cos\theta + B_{s,y} sin\theta sin\phi + B_{s,z} sin\theta cos\phi \\ 0 + B_{s,y} cos\phi - B_{s,z} sin\phi \\ -B_{s,x} sin\theta +  B_{s,y} cos\theta sin\phi + B_{s,z} cos\theta cos\phi \end{bmatrix} 
\end{equation}

Eq.\ref{eq:yaw} is now derived using Eq.\ref{eq:mag_3}, where yaw angle $\psi$ represents the tilt compensated bearing of the vessel. The ATAN2() function in python is used to determining 
the yaw angle, it returns a value in the range of $-180^{\circ}$ to $180^{\circ}$. The following check is therefore done in software: if the yaw angle in negative then $360^{\circ}$ is added
to it. This ensures bearing is in the range $0^{\circ}$ to $360^{\circ}$.

\begin{equation}
    \label{eq:yaw}
    \psi =  \frac{-B_{r,y}}{B_{r,x}} \newline
        =\arctan (\frac{B_{s,z} sin\phi - B_{s,y} cos\phi }  {B_{s,x} cos\theta + B_{s,y} sin\theta sin\phi + B_{s,z} sin\theta cos\phi})
\end{equation}











\subsection{Calibration}

\subsection{Digital filter}


\section{Power System}
%include power consumption of all components here

\section{PCB}


